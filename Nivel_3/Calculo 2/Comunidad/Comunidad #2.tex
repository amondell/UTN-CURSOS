\documentclass{article}
  \usepackage[total={18cm,21cm},top=2cm, left=2cm]{geometry}
  %Paquetes adicionales
  \usepackage{latexsym,amsmath,amssymb,amsfonts}
  \usepackage[latin1]{inputenc}
  %\usepackage{txfonts}
  \usepackage[T1]{fontenc}\usepackage{ae,aecompl}
  \usepackage{graphicx}
  \usepackage[usenames]{color}
  \usepackage[spanish]{babel} % Idioma espa�ol
  \usepackage{tcolorbox}
\usepackage{multirow}
\tcbuselibrary{theorems}
  %---------------comandos especiales-------------------------------
  \newcommand{\R}{\mathbb{R}}
  \newcommand{\Z}{\mathbb{Z}}
  \newcommand{\Q}{\mathbb{Q}}
  \newcommand{\C}{\mathbb{C}}
  \newcommand{\N}{\mathbb{N}}
  \newcommand{\I}{\mathbb{I}}
  \newcommand{\F}{\mathbb{F}}
  %-----------------------------------------------------------------

\begin{document}


\begin{center}
\textbf{Comunidad aprendiente \#2}
\end{center}
 
\noindent {\sc Angie Marchena Mondell} \\
{\sc Christopher Torrentes Delgado} \\
{\sc Mar�a Mercedes rojas Alvarado} \\
{\sc Rodolfo Marten  Guidotti} \\ 
\begin{enumerate}
\item \textbf{Primer Problema}\\
\begin{enumerate}
\item $\displaystyle \{b_n\}=\frac{n}{n^2+1}$

\item Asumo que $b_{n+1} \leq b_n$
\begin{align*}
\frac{n+1}{(n+1)^2+1} &\leq  \frac{n}{n^2+1}\\
(n+1)(n^2+1)  &\leq n[((n+1)^2+1)]\\
n^3+n+n^2+1  &\leq n[(n^2+2n+2]\\
n^3+n^2+n+1  &\leq n^3+2n^2+2n\\
n^2+n+1  &\leq 2n^2+2n\\
1 &\leq n^2+n
\end{align*}
Como se ve $1 \leq n^2+n$ es correcto ya que $n\geq 1$ por lo que $ n^2+n$ siempre es mayor que 1\\
Se concluye que $b_{n+1} \leq b_n$ por lo que es \textbf{DECRECIENTE}.

\item Se determina el limite
\begin{align*}
& \lim_{n \to \infty} b_n\\
& \lim_{n \to} \frac{n}{n^2+1} = \frac{\infty}{\infty}\\
& \lim_{n \to} \frac{n}{n(n+1/n)} = \frac{1}{\infty+0} = 0 \\
\end{align*}
Concluimos que es convergente a 0.
\end{enumerate}
\newpage
\item \textbf{Segundo Problema}\\
Dividimos en dos la expresi�n:
$$\sum_{n=2}^\infty \frac{3}{n^2+3n} + \sum_{n=2}^\infty \frac{2^{n-1}}{3^{n+2}}$$
Empezamos por la primera serie:
$$\sum_{n=2}^\infty \frac{3}{n^2+3n}$$
Separamos en fracciones parciales:
\begin{align*}
\frac{3}{n^2+3n} &= \frac{3}{n(n+3)}\\
\frac{3}{n(n+3)} &= \frac{A}{n}+\frac{B}{n+3}\\
3 &= A(n+3)+Bn\\
\end{align*}
Si $n=0$
\begin{align*}
3 &= 3A+0\\
A &=1
\end{align*}
Si $n=-3$
\begin{align*}
3 &= 0+-3B\\
B &=-1
\end{align*}
Por lo que reescribimos la suma como:
$$\sum_{n=2}^\infty \left(\frac{1}{n} - \frac{1}{n+3} \right)$$
Como se puede ver es una serie telesc�pica con $a_n=\frac{1}{n}$
Tenemos como resultado:
\begin{align*}
\sum_{n=2}^\infty \left(\frac{1}{n} - \frac{1}{n+3} \right) &= a_1+a_2+a_3-\lim_{n \to \infty} a_{n+3}\\
\sum_{n=2}^\infty \left(\frac{1}{n} - \frac{1}{n+3} \right) &= \frac{1}{2}+\frac{1}{3}+\frac{1}{4}-0\\
\sum_{n=2}^\infty \left(\frac{1}{n} - \frac{1}{n+3} \right) &= \frac{13}{12}\\
S_1=\frac{13}{12}
\end{align*}
Para la segunda expresi�n tenemos:
\begin{align*}
\sum_{n=2}^\infty \frac{2^{n-1}}{3^{n+2}} &=\sum_{n=2}^\infty \frac{2^{n}\cdot 2^{-1}}{3^n\cdot 3^2}\\
&=\sum_{n=2}^\infty \left(\frac{2}{3}\right)^n\cdot\frac{1}{18}\\
&=\sum_{n=2}^\infty \left(\frac{2}{3}\right)^n\\
&=\frac{1}{18} \sum_{n=2}^\infty \left(\frac{2}{3}\right)^n\\
\end{align*}
Se puede ver que es una serie geom�trica con $r=2/3$ y se ve claro que $|r|\leq 1$.
Usamos la formula para averiguar la suma.
\begin{align*}
\sum_{n=k}^\infty r^n &= \frac{r^k}{1-r}\\
\sum_{n=2}^\infty \left(\frac{2}{3}\right)^n &= \frac{(2/3)^2}{1-2/3}\\
\sum_{n=2}^\infty \left(\frac{2}{3}\right)^n &= \frac{4}{3}\\
\end{align*}
Por lo que tenemos:
\begin{align*}
S_2&=\frac{1}{18} \sum_{n=2}^\infty \left(\frac{2}{3}\right)^n\\
S_2&=\frac{1}{18}\cdot \frac{4}{3}\\
S_2&= \frac{2}{27}
\end{align*}
Finalizamos sumando las dos partes:
\begin{align*}
S&=S_1+S_2\\
S&=\frac{13}{12}+\frac{2}{27}\\
S&=\frac{125}{108}
\end{align*}
\newpage
\item \textbf{Tercer Problema}\\
\begin{enumerate}
\item $\displaystyle \sum_{n=1}^\infty \frac{n}{2n^3+1}$\\
Podemos realizar una comparaci�n en el limite de la siguiente manera:\\
Como tenemos que $1 \ll 2n^3$.
\begin{align*}
\frac{n}{2n^3+1} \sim  \frac{n}{2n^3} = \frac{1}{2n^2}
\end{align*}
Podemos usar como serie $\displaystyle \sum_{n=1}^\infty \frac{1}{2n^2}$
Que como sabemos $\displaystyle \sum_{n=1}^\infty \frac{1}{2n^2}$ converge ya que es una $p$ serie con $p>1$.\\
Realizamos el limite:

\begin{align*}
\lim_{n \to \infty} \frac{\frac{n}{2n^3-1}}{\frac{1}{2n^2}} =& \lim_{n \to \infty} \frac{n\cdot 2n^2}{2n^3-1}\\
=& \lim_{n \to \infty} \frac{ 2n^3}{2n^3-1}\\
=& 1
\end{align*}
Como el resultado del limite es un numero positivo determinamos que la serie \textbf{CONVERGE}.\\
\item $\displaystyle \sum_{n=1}^\infty \frac{\cos^2 (n)+1}{n^3}$\\

Para este caso podemos usar comparaci�n de la siguiente manera:
\begin{align*}
-1 \leq& \cos (n) \leq 1\\ 
0 \leq& \cos^2 (n) \leq 1\\ 
1 \leq& \cos^2 (n) +1 \leq 2\\
\frac{1}{n^3} \leq& \frac{\cos^2 (n) +1}{n^3} \leq \frac{2}{n^3}\\
\end{align*}
Si tomamos como $\displaystyle a_n = \frac{\cos^2 (n)+1}{n^3}$ y $\displaystyle b_n = \frac{2}{n^3}$
Verificamos la convergencia de $b_n$.
\begin{align*}
\sum_{n=1}^\infty \frac{2}{n^3} &= 2\sum_{n=1}^\infty \frac{1}{n^3} \\
\end{align*}
Como vemos es una $p$ serie con $p\geq 1$ por lo que si converge, y a su vez $a_n\leq b_n$ por lo que se concluye por \textbf{comparaci�n directa} que $ a_n = \frac{\cos^2 (n)+1}{n^3}$ \textbf{CONVERGE}.
\clearpage
\item $\displaystyle \sum_{k=1}^\infty \left( \frac{k+1}{2k+1} \right)^{2k-1}$\\
En este caso podemos usar el criterio de la \textbf{ra�z}, de la siguiente manera:
\begin{align*}
\lim_{k \to \infty} \sqrt[k]{\left|\left( \frac{k+1}{2k+1} \right)^{2k-1}\right|} &= \lim_{k \to \infty} \sqrt[k]{\left( \frac{k+1}{2k+1} \right)^{2k}\cdot \left(\frac{k+1}{2k+1} \right)^{-1}} \\
&= \lim_{k \to \infty}\left( \frac{k+1}{2k+1} \right)^{2}\cdot \sqrt[k]{\left(\frac{k+1}{2k+1} \right)^{-1}} \\
&= \lim_{k \to \infty}\left( \frac{k+1}{2k+1} \right)^{2}\cdot \left(\frac{k+1}{2k+1} \right)^{-1/k}\\
&= \left( \frac{1}{2} \right)^{2}\cdot \left(\frac{1}{2} \right)^{0}\\
&= \frac{1}{4} 
\end{align*}
Como obtenemos $\frac{1}{4}$ vemos que $\frac{1}{4}<1$ por lo que la serie \textbf{CONVERGE}.\\

\item $\displaystyle \sum_{n=1}^\infty \frac{5^n\cdot n!}{2\cdot5\cdot8\cdot \cdot \cdot (3n-1)}$\\
Utilizando el criterio del cociente:
\begin{align*}
\lim_{n \to \infty} \frac{a_{n+1}}{a_n} &= \lim_{n \to \infty} \left| \frac{\frac{5^{n+1}\cdot (n+1)!}{2\cdot5\cdot8\cdot \cdot \cdot (3n-1)(3(n+1)-1)}}{\frac{5^n\cdot n!}{2\cdot5\cdot8\cdot \cdot \cdot (3n-1)}}\right|\\
&= \lim_{n \to \infty} \frac{\frac{5^{n} \cdot 5\cdot n!(n+1)}{2\cdot5\cdot8\cdot \cdot \cdot (3n-1)(3n+2)}}{\frac{5^n\cdot n!}{2\cdot5\cdot8\cdot \cdot \cdot (3n-1)}}\\
&= \lim_{n \to \infty} \frac{\frac{5(n+1)}{(3n+2)}}{\frac{1}{1}}\\
&= \lim_{n \to \infty} \frac{5n+5}{3n+2}\\
&=\frac{5}{3}\\
\end{align*}
Como se puede ver el valor del limite es $\frac{5}{3}$ que es mayor que 1, por lo que concluimos por el criterio del cociente que $\displaystyle \sum_{n=1}^\infty \frac{5^n\cdot n!}{2\cdot5\cdot8\cdot \cdot \cdot (3n-1)}$  \textbf{DIVERGE}.
\end{enumerate}
\end{enumerate}
\end{document}