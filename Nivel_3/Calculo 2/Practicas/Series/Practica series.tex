\documentclass{article}
  \usepackage[total={18cm,21cm},top=2cm, left=2cm]{geometry}
  %Paquetes adicionales
  \usepackage{latexsym,amsmath,amssymb,amsfonts}
  \usepackage[latin1]{inputenc}
  %\usepackage{txfonts}
  \usepackage[T1]{fontenc}\usepackage{ae,aecompl}
  \usepackage{graphicx}
  \usepackage[usenames]{color}
  \usepackage[spanish]{babel} % Idioma espa�ol
  \usepackage{tcolorbox}
\usepackage{multirow}
\tcbuselibrary{theorems}
  %---------------comandos especiales-------------------------------
  \newcommand{\R}{\mathbb{R}}
  \newcommand{\Z}{\mathbb{Z}}
  \newcommand{\Q}{\mathbb{Q}}
  \newcommand{\C}{\mathbb{C}}
  \newcommand{\N}{\mathbb{N}}
  \newcommand{\I}{\mathbb{I}}
  \newcommand{\F}{\mathbb{F}}
  %-----------------------------------------------------------------
\usepackage{fancyhdr}
\pagestyle{fancy}
\fancyfoot[R]{Emanuel Esquivel L.}
\begin{document}


{\sc Calculo II} 

{\sc Emanuel Esquivel L.} \hfill   
\begin{center}
{\sc \textbf{Practica: Series}}\\
\end{center}
Determine la si las siguientes series divergen o convergen, si converge determine la suma.

\begin{enumerate}
\item $\displaystyle \sum_{n=1}^{\infty}  \frac{1+2^n}{3^n} $

\item $\displaystyle \sum_{n=1}^{\infty}   \frac{2 \cdot 3^n}{4^n} $

\item $\displaystyle \sum_{n=1}^{\infty}  \frac{2 }{n^2-1} $

\item $\displaystyle \sum_{k=3}^{\infty}  \frac{4^{k+1}}{5^k}$

\item $\displaystyle \sum_{n=1}^{\infty}  \frac{-1}{n^2+3n+2} $
\end{enumerate}


\noindent \textbf{Extras}
\begin{enumerate}
\item Si se sabe que  $\displaystyle \sum_{n=1}^{\infty}  \frac{1}{n^2}=\frac{\pi^2}{6}$ calcule:
$$\sum_{n=1}^{\infty}  \frac{1}{n^2(n+1)^2}$$

\item Determine la suma de la siguiente serie:
$$\sum_{n=2}^{\infty} \left(  \frac{4}{n(n+1)} + \frac{(-1)^{n+1}}{5^n}\right)$$
\end{enumerate}
\end{document}