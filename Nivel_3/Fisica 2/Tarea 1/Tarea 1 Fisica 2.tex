\documentclass[letterpaper,12pt,addpoints]{exam}
\usepackage[utf8]{inputenc}
\usepackage[spanish]{babel}

\usepackage[top=1in, bottom=1in, left=0.75in, right=0.75in]{geometry}
\usepackage{amsmath,amssymb}
\usepackage{parskip}

\usepackage{varwidth}
\usepackage{graphicx}
\usepackage{xcolor}
\usepackage{tcolorbox}
\tcbuselibrary{theorems}


\pagestyle{headandfoot}
\firstpageheader{}{}{}
\firstpagefooter{}{Página \thepage\ de \numpages}{}
\runningheadrule
\runningfooter{}{Página \thepage\ de \numpages}{}
\definecolor{ceruleanblue}{rgb}{0.16, 0.32, 0.75}
\begin{document}
 
Emanuel Esquivel Lopez - 2016133597\\
\textbf{Problema 1:}\\

{\color{ceruleanblue}
Solución:\\
\begin{enumerate}
\item[a] Para calcular el periodo es necesario calcular la frecuencia angular $\omega$:\\
\begin{minipage}[t]{0.5\textwidth}
\begin{align*}
\omega &= \sqrt{\frac{k}{m}}\\
\omega &= \sqrt{\frac{25}{1}}\\
\omega &= 5\,\,\mathrm{rad/s}\\
\end{align*}
\end{minipage}
\begin{minipage}[t]{0.5\textwidth}
\begin{align*}
T &= \frac{2\pi}{\omega}\\
T &= \frac{2\pi}{5}
\end{align*}
\end{minipage}
\begin{center}
\fcolorbox{ceruleanblue}{white}{$T = 1.256\,\,\mathrm{s}$}
\end{center}

\item[b] Calculo de valores máximos\\
\begin{minipage}[t]{0.5\textwidth}
\begin{align*}
v_{\max} &= A\omega\\
v_{\max} &=(0.03) \cdot (0.5)
\end{align*}
\begin{center}
\fcolorbox{ceruleanblue}{white}{$v_{\max} = 0.15\,\,\mathrm{m/s}$}
\end{center}

\end{minipage}
\begin{minipage}[t]{0.5\textwidth}
\begin{align*}
a_{\max} &= A\omega^2\\
a_{\max} &=(0.03) \cdot (0.5)^2
\end{align*}
\begin{center}
\fcolorbox{ceruleanblue}{white}{$a_{\max} = 0.75\,\,\mathrm{m/s}^2$}
\end{center}
\end{minipage}




\item[c] Posición velocidad y aceleración en función del tiempo
\begin{align*}
x(t) &= -A\cos(\omega t)
\end{align*}
\begin{center}
\fcolorbox{ceruleanblue}{white}{$x(t) = -3\cos(5t)\,\,\mathrm{cm}$}
\end{center}

\begin{minipage}[t]{0.5\textwidth}
\begin{align*}
v(t) &= \frac{dx}{dt}
\end{align*}
\begin{center}
\fcolorbox{ceruleanblue}{white}{$v(t) = 15\sin(5t)\,\,\mathrm{cm/s}$}
\end{center}

\end{minipage}
\begin{minipage}[t]{0.5\textwidth}
\begin{align*}
a(t) &= \frac{dv}{dt}
\end{align*}
\begin{center}
\fcolorbox{ceruleanblue}{white}{$a(t) = 0.75\cos(5t)\,\,\mathrm{cm/s}^2$}
\end{center}
\end{minipage}


\end{enumerate}
}
\clearpage
\textbf{Problema 2:}

{\color{ceruleanblue}
Solución:\\
\begin{enumerate}
\item[a] Calculamos la frecuencia angular
\begin{align*}
\omega &= \sqrt{\frac{k}{m}}\\
\omega &= \sqrt{\frac{8}{0.5}}\\
\omega &= 4\,\,\mathrm{rad/s}\\
\end{align*}
Con esto podemos determinar $x(t)$
\begin{align*}
x(t) &= 10\sen(4 t)\,\,\mathrm{cm}
\end{align*}

\begin{minipage}[t]{0.5\textwidth}
\begin{align*}
v(t) &= \frac{dx}{dt}=40\cos(4t)\,\,\mathrm{cm/s}
\end{align*}
\begin{center}
\fcolorbox{ceruleanblue}{white}{$v_{\max} = 40\,\,\mathrm{cm/s}$}
\end{center}

\end{minipage}
\begin{minipage}[t]{0.5\textwidth}
\begin{align*}
a(t) &= \frac{dv}{dt}=-160\sen(4t)\,\,\mathrm{cm/s^2}
\end{align*}
\begin{center}
\fcolorbox{ceruleanblue}{white}{$a_{\max} = 160\,\,\mathrm{cm/s^2}$}
\end{center}
\end{minipage}

\item[b] Para calcular rapidez y aceleración cuando $x=6$ cm, calculamos $t$

\begin{align*}
6 &= 10\sen(4 t)\,\,\mathrm{cm}\\
t &= 0.160\,\,\mathrm{s}
\end{align*}

\begin{minipage}[t]{0.5\textwidth}
\begin{align*}
v(0.160) &=40\cos(4\cdot(0.160))\,\,\mathrm{cm/s}
\end{align*}
\begin{center}
\fcolorbox{ceruleanblue}{white}{$v(0.160) = 32.08\,\,\mathrm{cm/s}$}
\end{center}
\end{minipage}
\begin{minipage}[t]{0.5\textwidth}
\begin{align*}
a(0.160) &=-160\sen(4\cdot(0.160))\,\,\mathrm{cm/s^2}
\end{align*}
\begin{center}
\fcolorbox{ceruleanblue}{white}{$a(0.160) = -95.5\,\,\mathrm{cm/s^2}$}
\end{center}
\end{minipage}

\item[c] Como sabemos en $x=0$ el tiempo es 0, si calculamos el tiempo en 8 cm
\begin{align*}
8 &= 10\sen(4 t)\,\,\mathrm{cm}\\
t &= 0.231\,\,\mathrm{s}
\end{align*}
Por lo que obtenemos que $\Delta t$
\begin{center}
\fcolorbox{ceruleanblue}{white}{$\Delta t= 0.231\,\,\mathrm{s}$}
\end{center}
\end{enumerate}
}

\clearpage
\textbf{Problema 3:}


{\color{ceruleanblue}
Solución:\\
\begin{enumerate}
\item[a] Debemos tener la frecuencia en rad/s


\begin{minipage}[t]{0.5\textwidth}
\begin{align*}
\omega &= 0.5\,\,\mathrm{rev/s}\\
\omega &= \frac{0.5\,\,\mathrm{rev}}{1\,\,\mathrm{s}} \cdot\frac{2\pi\,\,\mathrm{rad}}{1\,\,\mathrm{rev}}\\
\omega &=\pi\,\,\mathrm{rad/s}
\end{align*}
\end{minipage}
\begin{minipage}[t]{0.5\textwidth}
\begin{align*}
T &= \frac{2\pi}{\omega}\\
T &= \frac{2\pi}{\pi}
\end{align*}
\end{minipage}
\begin{center}
\fcolorbox{ceruleanblue}{white}{$T = 2\,\,\mathrm{cm/s}$}
\end{center}
\item[b] Para determinar la frecuencia tenemos que:
\begin{align*}
f &=\frac{1}{T}\\
f &=\frac{1}{02}
\end{align*}
\begin{center}
\fcolorbox{ceruleanblue}{white}{$f = 0.5\,\,\mathrm{Hz}$}
\end{center}
\item[c] La amplitud A es
\begin{center}
\fcolorbox{ceruleanblue}{white}{$A = 30\,\,\mathrm{cm}$}
\end{center}
\item[d] Ecuación de desplazamiento en función del tiempo:
\begin{align*}
x(t) &= A\sen(\omega t \pm \phi) 
\end{align*}
\begin{center}
\fcolorbox{ceruleanblue}{white}{$x(t) = 0.3\sen(\pi t) \,\mathrm{m}$}
\end{center}


\end{enumerate}
}
\end{document}