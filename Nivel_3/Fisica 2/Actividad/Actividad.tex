\documentclass[letterpaper,12pt,addpoints]{exam}
\usepackage[utf8]{inputenc}
\usepackage[spanish]{babel}

\usepackage[top=1in, bottom=1in, left=0.75in, right=0.75in]{geometry}
\usepackage{amsmath,amssymb}
\usepackage{parskip}

\usepackage{varwidth}
\usepackage{graphicx}
\usepackage{xcolor}
\usepackage{tcolorbox}
\tcbuselibrary{theorems}




%\usepackage{concmath}
%\usepackage{kmath,kerkis}
%\usepackage{eulervm}%like 2
%\usepackage{sansmath}
%\usepackage{sfmath}
%\usepackage{sansmathaccent}
%\usepackage{mathastext}
%\usepackage{MnSymbol}
%\usepackage{ccfonts}
\usepackage{antpolt}%like 1
%\usepackage{fouriernc}%like 3

\usepackage[QX]{fontenc}%like 1
%\usepackage[T1]{fontenc}

\definecolor{ceruleanblue}{rgb}{0.16, 0.32, 0.75}
\begin{document}
 
Emanuel Esquivel Lopez - 2016133597\\
\textbf{Problema 1:}\\

{\color{ceruleanblue}
Solución:\\
\begin{enumerate}
\item[a)] Se puede ver que la fuerza de empuje tiene la forma:
$$\vec{F}_E = -\frac{mg}{d}y\,\hat{\jmath}$$
Similar a la formula de la fuerza restauradora $F = -k\textbf{r}$.
Por lo que se puede deducir que $$k = \frac{mg}{d}$$

\item[b)] Podemos ver que la segunda ley de Newton
$$\sum F_y = ma = -ky$$
Al usar esto podemos realizar el cambio de $a$ como derivada de la posición:
\begin{align*}
ma &= -ky \\
m\frac{dy^2}{dt^2} &= -ky\\
\frac{dy^2}{dt^2}+ky &= 0 \\
\frac{dy^2}{dt^2}+\frac{k}{m} y &= 0 \\
\frac{dy^2}{dt^2}+\frac{g}{d} y &= 0 
\end{align*}
Puede compararse con el movimiento armónico simple que tiene por ecuación:
$$\frac{dy^2}{dt^2}+ \omega^2 y = 0 $$

\item[c)] La frecuencia angulas puede obtenerse fácil haciendo una comparación entre la ecuación obtenida y la del MAS.
\begin{align*}
\omega^2 &= \frac{g}{d}\\
\omega &= \sqrt{\frac{g}{d}}
\end{align*}
Con esto podemos calcular la frecuencia $f$ y el periodo $T$:
\begin{align*}
f &= \frac{\omega}{2\pi}\\
f &= \frac{1}{2\pi}\sqrt{\frac{g}{d}}\\ 
T &= \frac{2\pi}{\omega}\\
T &= 2\pi \sqrt{\frac{d}{g}}\\
\end{align*}


\item[d)] Tenemos como datos iniciales $y_0$ = 1 cm y $v=0$ m/s\\
Calculamos $A$
\begin{align*}
A &= \sqrt{y_0^2+\frac{v_0^2}{\omega^2}}\\
A &= y_0
\end{align*}
Ahora el angulo de fase:
\begin{align*}
\phi &= \arctan\left( -\frac{v_0}{y_0\omega} \right)\\
\phi &= 0
\end{align*}
La ecuación de la posición que es la solución de la ecuación diferencial es:
\begin{align*}
y(t) &= A\cos(\omega t + \phi)\\
y(t) &= y_0\cos\left( \sqrt{\frac{g}{d}}t \right)
\end{align*}


La ecuación de la velocidad es $y'$ y la aceleración $y''$


\begin{align*}
v(t) &= -A\omega \sen(\omega t + \phi)\\
v(t) &= -y_0\sqrt{\frac{g}{d}} \sen\left( \sqrt{\frac{g}{d}}t \right)
\end{align*}
\begin{align*}
a(t) &= -A\omega^2 \cos(\omega t + \phi)\\
a(t) &= -y_0\frac{g}{d} \cos\left( \sqrt{\frac{g}{d}}t \right)
\end{align*}


\item[e)] Los valores máximos de posición, velocidad y aceleración son los siguientes:
\begin{align*}
x_{\max} &= A\\
x_{\max} &= |y_0|\\
v_{\max} &= A\omega\\
v_{\max} &= |y_0|\sqrt{\frac{g}{d}}\\
a_{\max} &= A\omega^2\\
a_{\max} &= |y_0|\frac{g}{d}
\end{align*}

\item[f)] Luego de $t=3$ s podemos calcular el valor de $y,v,a$.

\begin{align*}
y(3) &= y_0\cos\left( \sqrt{\frac{g}{d}}\cdot 3 \right)\\
v(3) &= -y_0\sqrt{\frac{g}{d}} \sen \left( \sqrt{\frac{g}{d}}\cdot 3 \right)\\
a(3) &= -y_0\frac{g}{d} \cos\left( \sqrt{\frac{g}{d}} \cdot 3 \right)
\end{align*} 
Para el valor de $y$ debemos tomar en cuenta que el corcho estaba sumergido 3.25 cm
\item[g)] La energía cinética se describe con la formula:
\begin{align*}
K &=\frac{1}{2}mv^2\\
K &=\frac{1}{2}m y_0^2 \frac{g}{d} \sen^2\left( \sqrt{\frac{g}{d}}t \right)
\end{align*}


La energía potencial:

\begin{align*}
U &=\frac{1}{2}kx^2\\
U &=\frac{1}{2} \frac{mg}{d} y_0^2\cos^2\left( \sqrt{\frac{g}{d}}t \right)
\end{align*}
Cuando $U = \frac{1}{2}K$?
\begin{align*}
U &= \frac{1}{2}K\\
\frac{1}{2} \frac{mg}{d} y_0^2\cos^2\left( \sqrt{\frac{g}{d}}t \right)&= \frac{1}{4}m y_0^2 \frac{g}{d}  \sen^2\left( \sqrt{\frac{g}{d}}t \right)\\
\cos^2\left( \sqrt{\frac{g}{d}}t \right)&= \frac{1}{2}   \sen^2\left( \sqrt{\frac{g}{d}}t \right)\\
\frac{\cos^2\left( \sqrt{\frac{g}{d}}t \right)}{\cos^2\left( \sqrt{\frac{g}{d}}t \right)} &= \frac{1}{2}\frac{\sen^2\left( \sqrt{\frac{g}{d}}t \right)}{\cos^2\left( \sqrt{\frac{g}{d}}t \right)}\\
1 &= \frac{1}{2}\tan^2 \left( \sqrt{\frac{g}{d}}t \right)\\
2 &= \tan^2 \left( \sqrt{\frac{g}{d}}t \right)\\
\sqrt{2} &= \tan \left( \sqrt{\frac{g}{d}}t \right)\\
\arctan \left( \sqrt{2} \right) &= \sqrt{\frac{g}{d}}t\\
t &= \arctan \left( \sqrt{2} \right) \cdot \sqrt{\frac{d}{g}}
\end{align*}
Con el valor anterior se evalua en $y\left(\arctan \left( \sqrt{2} \right) \cdot \sqrt{\frac{d}{g}}\right) $

\item[h)] Es el valor de $t$ obtenido anteriormente.
$$t = \arctan \left( \sqrt{2} \right) \cdot \sqrt{\frac{d}{g}}$$

\end{enumerate}

\textbf{NOTA:} El valor de $y_0$ en las ecuaciones es negativo ya que esta pos debajo del equilibrio
}

\clearpage

\textbf{Problema 1:}\\

{\color{ceruleanblue}
Solución:\\
\begin{enumerate}
\item[a)] Al tener la función de onda $y(t,x) = A \sen(kx \pm \omega t)$, podemos comprobar que es solucion de la ecuación de onda de la siguiente manera.
\begin{align*}
y &= A \sen(kx \pm \omega t)\\
\frac{\partial y}{\partial x} &= Ak\cos(kx \pm \omega t)\\
\frac{\partial^2 y}{\partial x^2} &= -Ak^2\sen(kx \pm \omega t)
\end{align*}
\begin{align*}
y &= A \sen(kx \pm \omega t)\\
\frac{\partial y}{\partial t} &= A\omega \cos(kx \pm \omega t)\\
\frac{\partial^2 y}{\partial t^2} &= -A\omega^2\sen(kx \pm \omega t)\\
\frac{\partial^2 y}{\partial t^2} &= -Av^2k^2\sen(kx \pm \omega t)\\
\end{align*}
Si sustituimos en la ecuación de onda tenemos:
\begin{align*}
\frac{\partial^2 y}{\partial t^2} &= v^2 \frac{\partial^2 y}{\partial x^2} \\
-Av^2k^2\sen(kx \pm \omega t) &= v^2 \cdot -Ak^2\sen(kx \pm \omega t)
\end{align*}
Como se ve son exactamente iguales a ambos lados, por lo tanto se comprueba que $y(t,x) = A \sen(kx \pm \omega t)$ es solución de la ecuación de onda
\end{enumerate}
}
\end{document}