\documentclass{article}
  \usepackage[total={18cm,21cm},top=2cm, left=2cm]{geometry}
  %Paquetes adicionales
  \usepackage{latexsym,amsmath,amssymb,amsfonts}
  \usepackage[latin1]{inputenc}
  %\usepackage{txfonts}
  \usepackage[T1]{fontenc}\usepackage{ae,aecompl}
  \usepackage{graphicx}
  \usepackage[usenames]{color}
  \usepackage[spanish]{babel} % Idioma espa�ol
  \usepackage{tcolorbox}
\usepackage{multirow}
\tcbuselibrary{theorems}
  %---------------comandos especiales-------------------------------
  \newcommand{\R}{\mathbb{R}}
  \newcommand{\Z}{\mathbb{Z}}
  \newcommand{\Q}{\mathbb{Q}}
  \newcommand{\C}{\mathbb{C}}
  \newcommand{\N}{\mathbb{N}}
  \newcommand{\I}{\mathbb{I}}
  \newcommand{\F}{\mathbb{F}}
  %-----------------------------------------------------------------
\usepackage{fancyhdr}
\pagestyle{fancy}
\fancyfoot[R]{Emanuel Esquivel L.}
\begin{document}


{\sc Calculo III} 

{\sc Emanuel Esquivel L.} \hfill   
\begin{center}
{\sc \textbf{Formulas importantes}}\\
\end{center}
\textbf{Longitud de una curva}\\
Consideremos la curva parametrica $C(t)$ definida como:
$$C(t)=\left\{\begin{matrix}
x(t) & \\ 
y(t) & \\ 
z(t) & 
\end{matrix}\right.$$
con $t\in[a,b]$\\
La longitud de la curva esta dada por:
\begin{align*}
\int_a^b \sqrt{(x'(t))^2+(y'(t))^2+(z'(t))^2}\,dt
\end{align*}

\noindent \textbf{Recta tangente}\\
Consideremos la curva parametrica $C(t)$ definida como:
$$C(t)=\left\{\begin{matrix}
x(t) & \\ 
y(t) & \\ 
z(t) & 
\end{matrix}\right.$$
con $t\in[a,b]$, y un punto de tangencia $P(x_0,y_0,z_0)$\\
La recta tangente a la curva $C$ en $P$ se deduce como:
$$r(t)=\left\{\begin{matrix}
x &=& x_0+t\cdot x'(t_{x}) \\ 
y &=& y_0+t\cdot y'(t_y)\\ 
z &=& z_0+t\cdot z'(t_z)
\end{matrix}\right.$$ 
Donde el valor de $t_x, t_y$ y $t_z$ se averigua resolviendo las ecuaciones:
$$\left\{\begin{matrix}
x_0&=& x(t_x) \\ 
y_0 &=& y(t_y)\\ 
z_0 &=& z(t_z)
\end{matrix}\right.$$ 
\end{document}